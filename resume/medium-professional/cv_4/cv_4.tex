%%%%%%%%%%%%%%%%%%%%%%%%%%%%%%%%%%%%%%%%%
% Medium Length Professional CV
% LaTeX Template
% Version 2.0 (8/5/13)
%
% This template has been downloaded from:
% http://www.LaTeXTemplates.com
%
% Original author:
% Trey Hunner (http://www.treyhunner.com/)
%
% Important note:
% This template requires the resume.cls file to be in the same directory as the
% .tex file. The resume.cls file provides the resume style used for structuring the
% document.
%
%%%%%%%%%%%%%%%%%%%%%%%%%%%%%%%%%%%%%%%%%

%----------------------------------------------------------------------------------------
%	PACKAGES AND OTHER DOCUMENT CONFIGURATIONS
%----------------------------------------------------------------------------------------

\documentclass{resume} % Use the custom resume.cls style

\usepackage[left=0.75in,top=0.6in,right=0.75in,bottom=0.6in]{geometry} % Document margins

\name{Qiyuan Qiu} % Your name
\address{408 Quinby \\ Rochester, NY 14623} % Your address
\address{(585)~$\cdot$~747~$\cdot$~3721 \\ qiuqiyuan@gmail.com} % Your phone number and email

\begin{document}

%----------------------------------------------------------------------------------------
%	OBJECTIVE SECTION
%----------------------------------------------------------------------------------------
\begin{rSection}{Objective}
Seeking internship position in Computer Science with emphasis on distributed and parallel computing.
\end{rSection}


%----------------------------------------------------------------------------------------
%	EDUCATION SECTION
%----------------------------------------------------------------------------------------

\begin{rSection}{Education}
{\bf University of Rochester} \hfill {\em present}\\ 
PhD candidate in Computer Science \\ 
{\bf University of Minnesota, Twin Cities} \hfill {\em June 2013} \\ 
B.E. in Electrical Engineering \& Computer Science 
\end{rSection}

%----------------------------------------------------------------------------------------
%	WORK EXPERIENCE SECTION
%----------------------------------------------------------------------------------------

\begin{rSection}{Experience}

\begin{rSubsection}{University of Rochester}{October 2013 - Present}{Research Assistant}{Rochester, NY}
\item Design and implement message layer for parallel genome reconstruction algorithm.
\item Profile MPI program.
\item Nam tincidunt congue enim, ut porta lorem Microsoft SQL lacinia consectetur.
\item Donec ut libero sed arcu vehicula ultricies a non tortor. Lorem ipsum dolor sit amet, consectetur 
\end{rSubsection}

%------------------------------------------------

\begin{rSubsection}{Minnesota Supercomputing Institute}{March 2012 - June 2012}{Developer}{Minneapolis, MN}
\item Revised and implemented a newly invented algorithm on GPU platform.
\item Implemented parallel version code on multi-core CPU leveraging OpenMP and Pthread.
\item Offered novel solution to experiment design.
\item Led to a conference publication in the 24th International Symposium on Computer Architecture and High Performance Computing (SBAC-PAD).
\end{rSubsection}

%------------------------------------------------

\begin{rSubsection}{Seagate Technology}{June 2012 - Sep 2012}{Firmware Developer}{Shakopee, MN}
\item Debugged hard drive firmware source code over 1.8 million lines with the help of Coverity in multiple subsystems.
\item Processed massive data leveraging Google script language.
\item Collaborated with engineers remotely to resolve Coverity issues.
\item Offered detailed descriptions and possible solutions to defects in hard drive firmware source code.
\item Provided critical advice to help design an efficient algorithm resulting in the speed up web application.
\end{rSubsection}

\end{rSection}


%----------------------------------------------------------------------------------------
%	PUBLICATION
%----------------------------------------------------------------------------------------
\begin{rSection}{PUBLICATION}
J. Hu, Z. Wang, Q. Qiu, W. Xiao, and D. Lilja, “\textit{Sparse Fast Fourier Transform on GPUs and Multi-core CPUs,}” the 24th International Symposium on Computer Architecture and High Performance Computing (SBAC-PAD), Oct. 24- 26, 2012.
\end{rSection}

%----------------------------------------------------------------------------------------
%	AWARD & HONORS
%----------------------------------------------------------------------------------------
\begin{rSection}{AWARD \& HORNOR}
{\bf International Collegiate Programming Regional Contest}, $3^{rd}$ out of 239 \hfill Nov, 2012\\
{\bf International Collegiate Programming Regional Contest}, $5^{th}$ out of 215  \hfill Nov, 2011\\
{\bf Scholarship of Beijing Jiaotong University}, Creation Scholarship (1\%)  \hfill Sep, 2010\\
{\bf Beijing Physics Experiment Design Contest}, Gold Medal (1\%) \hfill Sep, 2010

\end{rSection}

%----------------------------------------------------------------------------------------
%	TECHNICAL STRENGTHS SECTION
%----------------------------------------------------------------------------------------

\begin{rSection}{Technical Strengths}

\begin{tabular}{ @{} >{\bfseries}l @{\hspace{6ex}} l }
Computer Languages & C/C++, Python, Lisp, JavaScript, Java \\
Operating Systems & Linux, Windows 
\end{tabular}

\end{rSection}

%----------------------------------------------------------------------------------------
%	EXAMPLE SECTION
%----------------------------------------------------------------------------------------

%\begin{rSection}{Section Name}

%Section content\ldots

%\end{rSection}

%----------------------------------------------------------------------------------------

\end{document}
