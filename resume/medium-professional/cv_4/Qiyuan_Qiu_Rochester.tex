%%%%%%%%%%%%%%%%%%%%%%%%%%%%%%%%%%%%%%%%%
% Medium Length Professional CV
% LaTeX Template
% Version 2.0 (8/5/13)
%
% This template has been downloaded from:
% http://www.LaTeXTemplates.com
%
% Original author:
% Trey Hunner (http://www.treyhunner.com/)
%
% Important note:
% This template requires the resume.cls file to be in the same directory as the
% .tex file. The resume.cls file provides the resume style used for structuring the
% document.
%
%%%%%%%%%%%%%%%%%%%%%%%%%%%%%%%%%%%%%%%%%

%----------------------------------------------------------------------------------------
%	PACKAGES AND OTHER DOCUMENT CONFIGURATIONS
%----------------------------------------------------------------------------------------

\documentclass{resume} % Use the custom resume.cls style

\usepackage[left=0.2in,top=0.3in,right=0.2in,bottom=0.3in]{geometry} % Document margins

\name{Qiyuan Qiu} % Your name
\address{408 Quinby \\ Rochester, NY 14623} % Your address
\address{(585)~$\cdot$~747~$\cdot$~3721 \\ qqiu@cs.rochester.com} % Your phone number and email

\begin{document}

%----------------------------------------------------------------------------------------
%	OBJECTIVE SECTION
%----------------------------------------------------------------------------------------
\begin{rSection}{Objective}
Seeking summer internship position utilizing my progrmming skill.  
\end{rSection}


%----------------------------------------------------------------------------------------
%	EDUCATION SECTION
%----------------------------------------------------------------------------------------

\begin{rSection}{Education}
{\bf University of Rochester} \hfill {\em present}\\
PhD candidate in Computer Science \\ 
{\bf University of Minnesota, Twin Cities} \hfill {\em June 2013} \\ 
B.E. in Electrical \& Computer Engineering
\end{rSection}

%----------------------------------------------------------------------------------------
%	WORK EXPERIENCE SECTION
%----------------------------------------------------------------------------------------

\begin{rSection}{RELATED Experience}

\begin{rSubsection}{Research Assistant - University of Rochester}{October 2013 - Present}{}{}
\item Designed and implemented message parsing, queuing, and dispatching system for large distributed system using MPI using C/C++.
\item Researched and studied Graphlab, a parallel asynchronous processing runtime for large graph structure. 
\item Researched and studied the stage-of-art parallel genome assembly software ABYSS, including profiling program performance using Linux tools, processing large amounts log using Python. 
\item Designed and exercised large number of experiments using SLURM on IBM Blue Gene cluster. 
\end{rSubsection}

%------------------------------------------------
\begin{rSubsection}{Course Project Team Leader - University of Rochester}{Sep 2014 - Dec 2014}{}{}
\item Process large amount of log file using hadoop. 
\item Annotated CPython implementation and carefully traced Python bytecode for function calls. 
\item Used Graphlab, a parallel graph processing runtime, to process large log file of Pythontutor, the largest online Python source visualization site.
\item Set up a website as the final project presentation using JavaScript, PHP and SQLite. 
\end{rSubsection}

%------------------------------------------------

\begin{rSubsection}{Developer - Minnesota Supercomputing Institute}{Jan 2012 - May 2012}{}{}
\item Revised and implemented a newly invented algorithm on GPU clusters as part of 3 person team.
\item Ran experiments on GPU clusters and processed result from large numbers of log using Python.
\item Implemented parallel version code on multi-core CPU leveraging OpenMP and Pthread.
\end{rSubsection}


%-------------------------------------------------
\begin{rSubsection}{International Collegiate Programming Contest}{Sep 2011 - Dec 2012}{}{}
\item Designed and implemented solutions algorithm problems as part of a 3 person team, responsible mostly for graph problems (network flow in particular) .
\item Ranked top 5 two years in a row. 
\end{rSubsection}

\end{rSection}


%----------------------------------------------------------------------------------------
%	PUBLICATION
%----------------------------------------------------------------------------------------
\begin{rSection}{PUBLICATION}
J. Hu, Z. Wang, Q. Qiu, W. Xiao, and D. Lilja, “\textit{Sparse Fast Fourier Transform on GPUs and Multi-core CPUs,}” the 24th International Symposium on Computer Architecture and High Performance Computing (SBAC-PAD), Oct. 24- 26, 2012.
\end{rSection}

%----------------------------------------------------------------------------------------
%	AWARD & HONORS
%----------------------------------------------------------------------------------------
\begin{rSection}{AWARD \& HORNOR}
{\bf International Collegiate Programming Regional Contest}, $3^{rd}$ out of 239 \hfill Nov, 2012\\
{\bf International Collegiate Programming Regional Contest}, $5^{th}$ out of 215  \hfill Nov, 2011\\
{\bf Tau Beta Pi Engineering Honor Society Minnesota Chapter Alpha}\hfill June 2011 \\
{\bf Scholarship of Beijing Jiaotong University}, Creation Scholarship (1\%)  \hfill Sep, 2010\\
{\bf Beijing Physics Experiment Design Contest}, Gold Medal (1\%) \hfill Sep, 2010 
\end{rSection}

%----------------------------------------------------------------------------------------
%	STRENGTHS SECTION
%----------------------------------------------------------------------------------------

\begin{rSection}{Strengths}

\begin{tabular}{ @{} >{\bfseries}l @{\hspace{6ex}} l }
Computer Languages & C/C++, Python, Lisp, JavaScript, Java (Hadoop), SQL \\
Courses Taken & Parallel \& Distributed System, GPU Computing, Dynamic Language Design \\
&Computer Architecture, Digital System Design, Operating System, Machine Learning
\end{tabular}

\end{rSection}

%----------------------------------------------------------------------------------------
%	EXAMPLE SECTION
%----------------------------------------------------------------------------------------

%\begin{rSection}{Section Name}

%Section content\ldots

%\end{rSection}

%----------------------------------------------------------------------------------------

\end{document}
